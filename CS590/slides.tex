\documentclass{beamer}
\usepackage{pgfpages}
\usepackage{complexity}

\usetheme{Madrid}
\useinnertheme{rectangles}
\usefonttheme{serif}

\title{Linial-Mansour-Nisan Theorem}
\subtitle{Crash Course on Fourier Analysis on Boolean Functions}
\author{Edward Kim}

\begin{document}

\begin{frame}
\titlepage
\end{frame}

\begin{frame}
\label{contents}
\frametitle{Outline}
\tableofcontents
\end{frame}

\section{Fourier Analysis on Boolean Functions}
\subsection{Characters}

\begin{frame}
\frametitle{Characters}
Let us first define some general concepts for any finite abelian group $G$.

\begin{enumerate}
\begin{definition}
  A group homomorphism from $\chi: G \rightarrow \mathbb{C}^*$ is called a \emph{character} of $G$ where $\mathbb{C}^* = \mathbb{C}/\{0\}$
\end{definition}

\item We call the homomorphism $\chi_0:G \rightarrow \mathbb{C}^*, \; \chi_0 = 1$ the \emph{trivial character} of $G$.
\item $\chi(a + b) = \chi(a)\chi(b)$
\item  The characters of $G$ form an abelian group $\hat{G}$ under pointwise multiplication of complex-valued functions. The group $\hat{G}$ is known as the \emph{character group} of $G$.
\end{enumerate}
\end{frame}

\begin{frame}
\frametitle{Characters}
  \begin{enumerate}
    \item It follows from basic properties of cyclic groups that the only characters of $\mathbb{Z}_n$ are ones of the form $$ \chi_j(x) = e^{ 2\pi i jx/n} \quad j \in [n], \; x \in \mathbb{Z}_n $$ Note that each character is associated to an element of the group $j \in \mathbb{Z}_n$
    \begin{theorem}[Characters for finite abelian groups]
    For any finite abelian group $G \cong \mathbb{Z}_{n_1} \times \mathbb{Z}_{n_2} \cdots \times\mathbb{Z}_{n_k}$: the characters for $G$ will be:
    $$   \chi_a(x) = \prod_{i \in [k]} e^{2 \pi i a_k x_k/ n_k} $$
    for $a \in \mathbb{Z}_{n_1} \times \mathbb{Z}_{n_2} \cdots \times\mathbb{Z}_{n_k}$
    \end{theorem}
  \end{enumerate}
\end{frame}

\begin{frame}
\frametitle{Characters}
  \begin{enumerate}
    \item It turns out the characters live in the vector space $L_2(G)$ i.e maps from $\phi: G \rightarrow \mathbb{C}$ square-integrable in respect to the uniform probability measure: $\frac{1}{|G|}\sum_{x \in G} |\phi(x)|^2 < \infty$
    \item Actually a Hilbert space endowed with the inner product:
    $$ \langle f,g \rangle = \frac{1}{|G|}\sum_{x\in G} f(x)\overline{g(x)}= \mathbb{E}_x f(x)\overline{g(x)} $$
    \begin{theorem}
      The characters $\chi \in \hat{G}$ form an orthonormal basis for $L_2(G)$
    \end{theorem}
    \item
    With any orthonormal basis, one can decompose any vector into its direct sum decomposition.
  \end{enumerate}
\end{frame}

\begin{frame}
\frametitle{Characters}
  \begin{enumerate}
    \item We restrict our attention to the finite abelian group $\mathbb{Z}_2^n$ as a natural group to define boolean functions $f: \{0,1\}^n \rightarrow \{0,1\}$.
    \item By the observations above, we conclude that our characters for $\mathbb{Z}_2^n = \mathbb{Z}_2 \times ... \times \mathbb{Z}_2$ are defined as
    $$  \chi_a(x) = (-1)^{\sum_{i \in [n]} x_ia_i} = (-1)^{\sum_{i \in [n], \; a_i = 1} x_i}$$
    where $x_i$ is the $i^{th}$ bit of $x \in \{0,1\}^n$.
    \begin{block}{Some Notation}
      Let $S \subseteq [n]$ be such that $S_x = \{i \mid x_i = 1\}$. As there is a bijective correspondence between all such subsets $S_x$ and $x \in \{0,1\}^n$, we will sometimes identify bit strings with their subset counterparts
      $$ \chi_A(x) = (-1)^{\sum_{i \in A} x_i}, \quad A \subseteq [n]$$
    \end{block}
  \end{enumerate}
\end{frame}


\subsection{The Fourier Transform}
\begin{frame}
\frametitle{The Fourier Transform}
  \begin{enumerate}
    \item Given any $f \in L_2(G)$, let $\hat{f}: \hat{G} \rightarrow \mathbb{C}$ be the complex-valued function such that
    $$  \hat{f}(\chi_a) =  \langle f, \chi_a \rangle, \; a \in G $$
    These are the projections onto the orthonormal basis of characters. We deem $\hat{f}$ the \emph{Fourier Transform} of $f$.
    \item The direct sum decomposition of $f$ yields the following form known as the \emph{Fourier Inversion Formula}:
    $$  f = \sum_{a \in G} \hat{f}(\chi_a)\chi_a $$
    The complex value $\hat{f}(\chi_a)$ is called the \emph{Fourier coefficient} associated to $\chi_a$
    \item From the above definition, we directly calculate that
    $$  \hat{f}(0) = \langle f, \chi_0 \rangle = \mathbb{E}_x f(x) $$
  \end{enumerate}
\end{frame}

\begin{frame}
\frametitle{The Fourier Transform}
\begin{enumerate}
  \item
  Let $f = \text{Maj}_3$ where $G = \mathbb{Z}_2^n$.
  \begin{align}
    & \hat{f}(0^3) = \mathbb{E}_x[f(x)] = \frac{1}{2} \\
    & \hat{f}(\{001,010,100\}) = - \frac{1}{4} \\
    & \hat{f}(\{011,110,101\}) = 0 \\
    & \hat{f}(1^3) = \frac{1}{8}\sum_{x \in \{0,1\}^3} (-1)^{|x|}f(x) = \frac{1}{4}
  \end{align}
  \begin{definition}
    Let $f:\{0,1\}^n \rightarrow \{0,1\}$ be an $n$-ary boolean function. The \emph{Fourier degree} of $f$, denoted by def$_{\mathcal{F}}(f)$, is the largest $|S|$ such that $\hat{f}(S) \neq 0$.
  \end{definition}
  For the case of f = Maj$_3$, def$_{\mathcal{F}}(f) = 3$
\end{enumerate}
\end{frame}

\begin{frame}
\frametitle{The Fourier Transform}
  \begin{theorem}[Parseval's Identity]
    Let $f \in L_2(G)$. Then
    $$ ||f||_2^2 = \sum_{a \in G} |\hat{f}(a)|^2$$
  \end{theorem}
  \begin{proof}
    \begin{align*}
      ||f||^2_2 = \langle f, f \rangle = \langle \sum_{a \in G} \hat{f}(a) \chi_a,  \sum_{b \in G} \hat{f}(b) \chi_b \rangle & = \sum_{a,b \in G} \hat{f}(a)\overline{\hat{f}(b)}\langle \chi_a, \chi_b \rangle \\
      & = \sum_{a \in G} |\hat{f}(a)|^2
    \end{align*}

    The last equality is just the orthonormality of the characters.
  \end{proof}
\end{frame}

\section{LMN Theorem}

\begin{frame}
  \frametitle{LMN Theorem}
  \begin{theorem} (Linial, Mansour, Nisan) \label{lmn}
      Let $f$ be a boolean fucntion computed by a circuit of depth $d$ and size $M$ and let $t$ be any non-negative integer. Then
      \begin{equation}
        \sum_{|S| > t} |\hat{f}(S)|^2 \leq 2M2^{-t^{1/d}/20}
      \end{equation}
  \end{theorem}
  The theorem reveals that the $t$-tails of the Fourier spectrum, i.e strings indexed by sets $|S| > t$, become exponentially small in $t$ for boolean functions in $\AC^0$.
\end{frame}

\begin{frame}
  \frametitle{LMN Theorem}

  \begin{theorem}[\emph{H\aa stad}]
    Let $f$ be given by a CNF-formula where each clause has size at most $t$, and choose a random restriction $\rho$ with parameter $p$ such that $Pr[\rho(x_i)] = p$ for all input variables $x_i$. With probability of at least $1 - (5pt)^s$, $f_{\rho}$ can be expressed as a DNF formula where each clause has size of at most $s$, and the clause all accept disjoint sets of inputs i.e no string $x \in \{0,1\}^n$ satisfies more than one clause.
  \end{theorem}

  \begin{corollary} \label{hastaddegree}
  Let $f$ be a boolean function computed by a CNF of bottom fan-in of at most $t$, and $\rho$ is a $p$-random restriction, then
  \begin{equation}
    Pr[\text{deg}_{\mathcal{F}}(f_\rho) > s] < (5pt)^s
  \end{equation}
  \end{corollary}

\end{frame}

\begin{frame}
  \frametitle{LMN Theorem}
  \begin{corollary}[Tail Degree Corollary] \label{fouriertail}
    Let $f$ be a boolean function computed by a circuit of size $M$ and depth $d$. Then
    $$ Pr[\text{deg}(f_{\rho}) > s] \leq M2^{-s} $$
    where $\rho$ is a random restriction where $p = \frac{1}{10^ds^{d-1}}$
  \end{corollary}
  \pause
  \begin{proof}[Proof Sketch]
    Show that first random restriction of parameter $p_0 = \frac{1}{10}$, the bottom gates' fan-ins are at most $s$ with probability of at least $1 - 2^{-s}$. Then iterate H\aa stad's switching lemma with under $p_i = \frac{1}{10s}$ on each gate of distance two fromm the input variables to turn them into DNFs with disjoint inputs. Collapse a level and the lemma ensures that the new bottom fan-in is at most $s$. Stop when we are left with a CNF (depth-2) with bottom fan-in of at most $s$ and invoke the previous corollary.
  \end{proof}
\end{frame}

\begin{frame}
  \frametitle{Proof of LMN Theorem}
  \begin{definition}
    Let $k$ be a positive integer and $f:\mathbb{Z}_2^n \rightarrow \mathbb{C}$ a complex-valued function on $\mathbb{Z}_2^n$. We define: \newline
    $$f^{\leq k} := \sum_{|S| \leq k} \hat{f}(S)\chi_{S} $$
    The notation symbols $f^{=k},f^{\geq k}$ are defined in the same manner.
  \end{definition}
  \pause
  \begin{enumerate}
    \item We begin by first setting our probability parameter $p \leq \frac{1}{10^dk^{d-1}}$. The values $p, k$ will be fixed later to invoke the Tail Degree Corollary above.
    \pause
    \item
    Recall how we sample a random restriction $\rho$ by sampling some $V \subseteq [n]$ to be the indices which are \emph{not} set to $*$. Each index has a $1-p$ probability of being set to either $0,1$. For each index in $V$, we uniformly sample some bit string in $\{0,1\}^{|V|}$ to fix the indices contained in $V$.
    \pause
    \item Let $x_{V}$ be the restriction of string $x$ on the input indices to those found in $V$
  \end{enumerate}
\end{frame}

\begin{frame}
  \frametitle{Proof of LMN Theorem}
  \begin{enumerate}
    \item If we recall that our characters are defined as $\chi_A(x) = (-1)^{\sum_{i \in A} x_i}$
    \pause
    \begin{block}{How characters separate}
      \begin{align*}
        \chi_S(x) = (-1)^{\sum_{i \in S} x_i}
        & = (-1)^{\sum_{i \in S \cap V} x_i + \sum_{i \in S / V} x_i}   \\
        & = (-1)^{\sum_{i \in S \cap V} x_i}(-1)^{\sum_{i \in S / V} x_i} = \chi_{S \cap V}(x_V) \chi_{S / V}(x_{\overline{V}})
      \end{align*}
    \end{block}
  \end{enumerate}
\end{frame}


\begin{frame}
  \frametitle{Proof of LMN Theorem}
  \begin{enumerate}
    \item If we set $x_V$ to some bit string, it makes sense to think of $f_{x_V} = f(x_V,*)$ as a function $f_{x_V}: \{0,1\}^{|x_{\overline{V}}|} \rightarrow \{0,1\}$.
    \pause
    \item By our Fourier Inversion Formula and the observation in the previous slide:
    \begin{align*}
    f(x)
    & = \sum_{S \subseteq [n]} \hat{f}(S)\chi_{S}(x) =  \sum_{V \sqcup \overline{V} \subseteq [n]} \hat{f}(S)\chi_{S \cap V}(x_V)\chi_{S / V}({x_{\overline{V}}}) \\
    & = \sum_{H \subseteq \overline{V}} \left( \sum_{J \subseteq V} \hat{f}(H \cup J) \chi_{J}(x_V) \right) \chi_H(x_{\overline{V}})
  \end{align*}
    \item
    Since the Fourier decomposition is unique for $f_{x_V}$,
    \begin{equation*}
      \widehat{f_{x_V}}(H) = \sum_{J \subseteq V} \hat{f}(H \cup J)\chi_{J}(x_V), \quad H \subseteq \overline{V}
    \end{equation*}
  \end{enumerate}
\end{frame}

\begin{frame}
  \frametitle{Proof of LMN Theorem}
  \begin{enumerate}
    \item By Parseval's identity on the function $x_T \mapsto \widehat{f_{x_T}}(H)$,
    \begin{equation}
      \mathbb{E}_{x_V}|\widehat{f_{x_V}}(H)|^2 = \langle \hat{f}_{-}(H), \hat{f}_{-}(H) \rangle = \sum_{J \subseteq V} |\hat{f}(H \cup J)|^2
    \end{equation}
    \pause
    \item
    This along with another application of Parseval's identity yields the string of equalities: \newline
    \begin{align}
      \mathbb{E}_{x_V}||f^{>k}_{x_V}||_2^2 = \mathbb{E}_{x_V}\sum_{\substack {H \subseteq \overline{V} \\ |H| > k}} |\widehat{f_{x_V}}(H)|^2
       & = \sum_{\substack {H \subseteq \overline{V} \\ |H| > k}} \sum_{J \subseteq V} |\hat{f}(H \cup J)|^2 \\
       & = \sum_{ \substack {S \subseteq [n] \\ |S \cap \overline{V}| > k}} |\hat{f}(S)|^2
    \end{align}
  \end{enumerate}
\end{frame}

\begin{frame}
  \frametitle{Proof of LMN Theorem}
  \begin{enumerate}
    \item Sampling over all such $p$-restrictions further shows that the lefthand side is upper bounded by:
    \begin{equation*} \label{coroinvoke}
      \mathbb{E}_V\mathbb{E}_{x_V} ||f^{>k}_{x_V}||_2^2 = \mathbb{E}_{\rho} ||f^{>k}_{\rho}||_2^2 \leq Pr[\text{deg}_{\mathcal{F}}(f_\rho) > k] \leq M2^{-k}
    \end{equation*}
    \pause
    \item A Chernoff bound argument shows that the righthand side is lower bounded by:
    $$ \sum_{|S| > t} |\hat{f}(S)|^2 \leq 2 \mathbb{E}_T \sum_{ \substack {S \subseteq [n] \\ |S \cap \overline{V}| > k}} |\hat{f}(S)|^2 $$
    for $k = pt/2$
    \pause
    \item Setting our constants $p = \frac{1}{10t^{(d-1)/d}}$ and $k = t^{1/d}/20$ shows that $p \leq \frac{1}{10^dk^{d-1}}$. This allows us to invoke our Tail Degree Corollary which gives us the desired inequality
    \begin{equation}
      \sum_{|S| > t} |\hat{f}(S)|^2 \leq 2M2^{-t^{1/d}/20}
    \end{equation}
  \end{enumerate}
\end{frame}

\section{Consequences}
\subsection{Approximation by Low-degree Polynomials}

\begin{frame}
  \frametitle{Approximation by Low-degree Polynomials}
   Theorem (\ref{lmn}) shows that we can approximate functions in $f \in \AC^0$ by taking our polynomials to be $f^{\leq k}$ for some sufficiently large $k$. Here we make the distinction that our Fourier expansion of $f$ will be
  $$f = \sum_{S \subseteq [n]} \hat{f}(S) \prod_{i \in S} x_i $$ as a polynomial with complex coefficients.
  \begin{lemma}
    Let $f \in \AC^0$ be a boolean function of polynomial size and depth $d$. Then there exists a complex polynomial of degree $\mathcal{O}((\log{n/\epsilon})^d)$ such that $||f - p ||_2 < \epsilon$
  \end{lemma}
\end{frame}

\subsection{Sensitivity and Influence}

\begin{frame}
  \frametitle{Sensitivity and Influence}
  LMN also shows that functions in $\AC^0$ have low average sensitivity i.e its output is not very sensitive to changes to the input.
  \begin{definition}
    Let $f: \{0,1\}^n \rightarrow \{0,1\}$ be a boolean function. We define the \emph{sensitivity} of an input $x\in \{0,1\}^n$ in respect to $f$, $s_f(x)$ to be the number of indices $i$ such that $f(x) \neq f(x + e_i)$ where $e_i$ is the bit string with zeros everywhere except for the $i^{th}$ index.
  \end{definition}

  \begin{definition}
    Let $f: \{0,1\}^n \rightarrow \{0,1\}$ be a boolean function. The \emph{influence} of $f$, $I_f$ is defined as the average sensitivity over all input bit strings
    \begin{equation}
      I_f = \mathbb{E}_x[s_f(x)]
    \end{equation}
  \end{definition}
\end{frame}

\begin{frame}
  \frametitle{Sensitivity and Influence}
  The influence of $f$ can be expressed in terms of its Fourier coefficients:

  \begin{equation}
    I_f = 4\sum_{S \subseteq [n]} |S||\hat{f}(s)|^2
  \end{equation}

  By combining this equivalence with Theorem (\ref{lmn}), we deduce the following upper bound on the influence of a function in $\AC^0$

  \begin{lemma} \label{influence}
    Let $f \in \AC^0$ be of depth $d$. Then
    \begin{equation}
      I_f = \mathcal{O}((\log n)^d)
    \end{equation}
  \end{lemma}
  The lemma shows that functions in $\AC^0$  are not suitable for constructing universal hash functions and pseudorandom function generators.
\end{frame}

\begin{frame}
  \frametitle{Sensitivity and Influence}
  \begin{definition}
    A function $f:\{0,1\}^m \times \{0,1\}^n \rightarrow \{0,1\}$ is called a \emph{pseudorandom function generator} if there exists no polynomial-time oracle Turing machine which can distinguish between the outputs from a true random oracle versus $f(s,_)$ for some random seed $s \in \{0,1\}^m$.
  \end{definition}

  \begin{lemma}
    No pseudorandom function generators exist in $\AC^0$
  \end{lemma}
  By taking advantage of the low average sensitivity, we can simply perturb the input slightly and check if the output of $f$ changes. If it doesn't, there is a good chance that
  it lies in $\AC^0$.
\end{frame}

\end{document}
