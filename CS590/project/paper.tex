\documentclass{amsart}
\usepackage{complexity}

\setlength{\textwidth}{\paperwidth}
\addtolength{\textwidth}{-2in}
\calclayout

\newtheorem{theorem}{Theorem}[section]
\newtheorem{lemma}[theorem]{Lemma}

\theoremstyle{definition}
\newtheorem{definition}[theorem]{Definition}
\newtheorem{example}[theorem]{Example}
\newtheorem{xca}[theorem]{Exercise}

\theoremstyle{remark}
\newtheorem{remark}[theorem]{Remark}
\numberwithin{equation}{section}

\theoremstyle{remark}
\newtheorem{notation}[theorem]{Notation}

\begin{document}

\title{On the Fourier Analysis of Boolean Functions and the Linai-Mansour-Nisan Theorem}

\author{Edward Kim}
\email{ehkim@cs.unc.edu}

\begin{abstract}
  We briefly summarize the basic tools of Fourier Analysis applied to the finite abelian group $\mathbb{Z}_2^n$. By viewing boolean functions $f:\{0,1\}^n \rightarrow \{0,1\}$ as square-integrable maps in the Hilbert space $L^2(\mathbb{Z}_2^n)$ under the uniform probability measure, we derive the Linai-Mansour-Nisan theorem concerning the tail of the Fourier spectrum of boolean functions in $\AC^0$. Finally, we highlight some applications of the LMN theorem in the context of sensitivity and influence of functions in $\AC^0$.
\end{abstract}

\maketitle

\section{Fourier Analysis of Boolean Functions}

\subsection{Characters}

We introduce some stepping stones defining the tools used in the Fourier Analysis of boolean functions. Many of these tools can be defined using techniques from Representation Theory, but for the sake of briefity, we will not delve into those concepts here in detail. Instead, we will provide remarks and snippets on relevant topics as needed along with references to the standard literature at the end of the section.  \newline

We first define some general concepts for any finite abelian group $G$. A group homomorphism from $\chi: G \rightarrow \mathbb{C}^*$ is called a \emph{character} of $G$ where $\mathbb{C}^* = \mathbb{C}/\{0\}$ is the multiplicative group of the complex numbers. We call the homomorphism $\chi_0:G \rightarrow \mathbb{C}^*, \; \chi_0 = 1$ the \emph{trivial character} of $G$. By definition, for $a,b \in G$,
\begin{equation} \label{homo}
\chi(a + b) = \chi(a)\chi(b)
\end{equation}
for all characters $\chi$. \newline

Consider the set whose elements are the characters of $G$ and whose binary operation is pointwise multiplication of complex-valued functions on $G$. This set is also an \emph{abelian group} by the following theorem:
%
\begin{theorem}
  If $G$ is a finite abelian group, then the characters of $G$ form an abelian group under pointwise multiplication.
\end{theorem}
%
\begin{proof}
  We first show that if $\chi, \phi$ are characters of $G$, then $\chi\phi$ is also a character of $G$. But this immediately follows from \ref{homo}:
  $$ \chi\phi(a+b) = \chi(a)\chi(b)\phi(a)\phi(b) = [\chi(a)\phi(a)][\chi(b)\phi(b)] = \chi\phi(a)\chi\phi(b)$$
  Hence, $\chi\phi$ is also a character of $G$. Furthermore, $\chi^{-1} = \frac{1}{\chi} = \overline{\chi}$ is also a character of $G$ since
  $$ \chi^{-1}(a+b) = \frac{1}{\chi(a+b)} = \frac{1}{\chi(a)\chi(b)} = \frac{1}{\chi(a)}\frac{1}{\chi(b)} = \chi^{-1}(a)\chi^{-1}(b)$$
  showing that the set has inverses.
\end{proof}
%
\begin{remark}
  We abuse notation a bit and just note that $\chi^{-1}$ is not the inverse of the $\chi$ as a \emph{map}. Rather, $\chi^{-1}$ will refer to the \emph{group inverse} of character $\chi$ from hereon.
\end{remark}

%
Denote this set as $\hat{G}$. The group $\hat{G}$ is known as the \emph{character group} of $G$. It also will turn out that there is a bijective map between $G$ and $\hat{G}$ such that the map is also a \emph{group isomorphism}. We will prove this later on. \newline

Let us now consider the simplest finite abelian group, $\mathbb{Z}_n = \mathbb{Z}/n\mathbb{Z}$ for some $n \in \mathbb{N}$. It follows from basic properties of cyclic groups that the only characters of $\mathbb{Z}_n$ are:
%
\begin{equation}
  \chi_j(x) = e^{ 2\pi i jx/n} \quad j \in [n], \; x \in \mathbb{Z}_n
\end{equation}
%
We can decompose any finite abelian group $G$ into a direct product of finite cyclic groups $G \cong \mathbb{Z}_{n_1} \times \mathbb{Z}_{n_2} \cdots \times\mathbb{Z}_{n_k}$. If we decompose $x = \sum_{i \in [k]} x_i$ for $x \in G$, then the characters of $G$ are defined as
%
\begin{equation}
  \chi_a(x) = \prod_{i \in [k]} e^{2 \pi i a_k x_k/ n_k}
\end{equation}
where $x_i$ refers to the $i^{th}$ index of $x$. \newline

We see that every element $a \in G$ has an unique associated character $\chi_a$. Furthermore
%
\begin{gather} \label{chargroup}
  \chi_{a+b}(x)  = \chi_{a}(x)\chi_{b}(x)
\end{gather} \newline

To define the Fourier transform, we must consider the Hilbert space $L_2(G)$ where $G$ endowed with the uniform probability measure i.e the discrete measure mapping each subset $H \subseteq G$ to $\frac{|H|}{|G|}$. This yields the standard inner product for maps $f,g \in L_2(G)$:
%
\begin{equation}
  \langle f,g \rangle = \frac{1}{|G|}\sum_{x\in G} f(x)\overline{g(x)}= \mathbb{E}_x f(x)\overline{g(x)}
\end{equation}
Note that this induces the norm $||f||_2^2 = \langle f, f \rangle = \mathbb{E}_x |f(x)|^2$ on $L_2(G)$. It is simple to check that the characters of $G$ indeed lie in $L_2(G)$. However, we can prove something stronger: the characters of $G$, $\chi_a$ form an \emph{orthonormal basis} of $L_2(G)$.
%
First, we prove a lemma:
\begin{lemma} \label{totalsum}
  Let $\chi$ be a non-trivial character of $G$. Then $\sum_{x \in G} \chi(x) = 0$
\end{lemma}
%
\begin{proof}
  Given any non-trivial character $\chi$, the following holds for all $y \in G$:
    %
    \[ \chi(y)\sum_{x \in G} \chi(x) = \sum_{x \in G} \chi(y+x) = \sum_{x \in G} \chi(x) \]
    %
    If $\sum_{x \in G} \chi(x) \neq 0$, then this would force $\chi(y) = 1$ for all $y \in G$. However, this would contradict the non-triviality of the $\chi$.
\end{proof}

\begin{theorem} \label{orthonormal}
  The characters $\chi \in \hat{G}$ form an orthonormal basis for $L_2(G)$
\end{theorem}
%
\begin{proof}
We first note that $\langle \chi, \chi \rangle = \mathbb{E}_x|\chi(x)|^2 = 1$ since the image of character $\chi$ lies on the complex unit circle by the discussions above. Now given two non-trivial characters $\chi, \phi$
$$ \langle \chi, \phi \rangle = \mathbb{E}_x \chi(x)\overline{\phi(x)} = \frac{1}{|G|} \sum_{x \in G} \chi\phi^{-1}(x)  = 0$$
The last equality follows from Lemma \ref{totalsum} since $\chi\phi^{-1}$ must also be a non-trivial character if $\chi, \phi$ are non-trivial characters. The case where one of the character is trivial also follows from the lemma.
Finally, observe that $L_2(G)$ is $|G|$-dimensional as a vector space by considering the dual of $G$. We know that there are at least $|G|$-characters, showing that the characters span $L_2(G)$. However, from the lemma, the characters must also be pairwise orthgonal to each other. This yields that the characters of $G$ form an \emph{orthonormal basis} of $L_2(G)$ and that there must be exactly $|G|$ characters in $\hat{G}$. Finally, by using \ref{chargroup}, it follows that the map $a \mapsto \chi_a$ must be a group isomorphism, so $G \cong \hat{G}$.
\end{proof}

To finish, we restrict our attention to the finite abelian group $\mathbb{Z}_2^n$ as a natural group to define boolean functions $f: \{0,1\}^n \rightarrow \{0,1\}$. By the observations above, we conclude that our characters for $\mathbb{Z}_2^n$ are defined as:
%
\begin{equation}
  \chi_a(x) = (-1)^{\sum_{i \in [n]} x_ia_i} = (-1)^{\sum_{i \in [n], \; a_i = 1} x_i}
\end{equation}
where $x_i$ is the $i^{th}$ bit of $x \in \{0,1\}^n$.
%
\begin{notation}
  Instead of considering $x \in \{0,1\}^n$ as a bit-string, we would like to use the following notation: Let $S \subseteq [n]$ be such that $S_x = \{i \mid x_i = 1\}$. As there is a bijective correspondence between all such subsets $S_x$ and $x \in \{0,1\}^n$, we will start to identify bit strings with their subset counterparts for notational convenience.
\end{notation}

\subsection{The Fourier Transform}

We have shown above that the members of the character group $\hat{G}$ form an orthonormal basis of $L_2(G)$. This bestows us the power of decomposing any $f \in L_2(G)$ into its orthogonal counterparts. The restriction to the case where $G = \mathbb{Z}_2^n$ is what motivates us as any $n$-ary boolean function $f: \{0,1\}^n \rightarrow \{0,1\}$ must lie in $L_2(\mathbb{Z}_2^n)$. As before, we begin by considering the general case for any finite abelian group $G$  \newline

Let $\hat{f}: \hat{G} \rightarrow \mathbb{C}$ be the complex-valued function defined as:
%
\begin{equation}
  \hat{f}(\chi_a) =  \langle f, \chi_a \rangle, \; a \in G
\end{equation}
%
We can iterate this for all characters to get a direct sum decomposition for any $f \in L_2(G)$
%
\begin{equation}
  f = \sum_{a \in G} \hat{f}(\chi_a)\chi_a
\end{equation}

It can be show that the map $f \mapsto \hat{f}$ is linear by the left-linearity of the inner product. This linear map is the \emph{Fourier Transform on $G$}.
%
\begin{notation}
  We will sometimes denote $\hat{f}(a) = \hat{f}(\chi_a)$.
\end{notation}


A simple application of the Fourier Transform yields an useful identity which will be instrumental in our formulation of the Fourier tail of a boolean function.

\begin{theorem}{(Parseval's Identity)}
  Let $f \in L_2(G)$. Then
  $$ ||f||_2^2 = \sum_{a \in G} |\hat{f}(a)|^2$$
\end{theorem}
%
\begin{proof}
  $$ ||f||^2_2 = \langle f, f \rangle = \langle \sum_{a \in G} \hat{f}(a) \chi_a,  \sum_{b \in G} \hat{f}(b) \chi_b \rangle = \sum_{a,b \in G} \hat{f}(a)\overline{\hat{f}(b)}\langle \chi_a, \chi_b \rangle = \sum_{a \in G} |\hat{f}(a)|^2 $$
  The last equality is just the orthonormality of the characters (\ref{orthonormal}).
\end{proof}

%by the remark in the previous section, we will identify characters of %$\mathbb{Z}_2^n$ as $\chi_A$ for $A \subseteq [n]$.


%\begin{remark}
%The above properties on characters can be shown through %techniques in Representation Theory. For instance, we %know representations of finite groups are group %homomorphisms into the group of linear isomorphisms of a %suitable complex vector space $\rho: G \rightarrow %GL(\mathbb{C}^n)$. The \emph{characters} of $\rho$ are %defined as $\chi(x) = Tr(\rho(x))$ where $Tr$ is the %trace operator. It turns out that the representations of %finite abelian groups are one-dimensional i.e isomorphic %to $\mathbb{C}$. Thus, $\rho(x)$ will just be multiplying %$\mathbb{C}$ by some scalar $\lambda_x \in \mathbb{C}^*$. %In this case, we can show that indeed for any $a,b \in G$ %and $\chi$ any character of $G$:
%$$ \chi(a+b) = Tr(\rho(a+b)) = Tr(\rho(a)\circ\rho(b)) = %\lambda_a \lambda_b = Tr(\rho(a))Tr(\rho(b)) = %\chi(a)\chi(b)$$ Note that this relation does not hold %for arbitrary finite groups. For more information on the %Representation Theory of Finite Groups, see \ref{serre}.
%\end{remark}





%\section{Linai-Mansour-Nisan Theorem}


%\section{Immediate Applications}





%    Bibliographies can be prepared with BibTeX using amsplain,
%    amsalpha, or (for "historical" overviews) natbib style.
\bibliographystyle{amsplain}
%    Insert the bibliography data here.

\end{document}
