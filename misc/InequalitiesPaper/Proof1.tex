\documentclass[11pt]{article}
\usepackage{amsmath,textcomp,amssymb,geometry,graphicx,enumerate}

\def\Name{Edward Kim}  % Your name

%for code%
\usepackage{listings}
\lstset{language=python}


\title{Proof of Inequality 1}
\author{\Name}

\textheight=9in
\textwidth=6.5in
\topmargin=-.75in
\oddsidemargin=0.25in
\evensidemargin=0.25in
\DeclareMathSizes{11}{19}{13}{9}


\begin{document}
\maketitle

\textbf{Proof:}
Let $a \geq 1$ and $b \geq 1$. We shall directly prove the following inequality.

\begin{equation}
\frac{1}{1+a} + \frac{1}{1+b} \geq \frac{2}{1 + \sqrt{ab}} 
\end{equation}

We first get the expanded version after multiplying both sides by all the terms. We notice that this preserves the inequality since all terms are positive.

\begin{equation}
(1+b)(1 + \sqrt{ab}) + (1+a)(1 + \sqrt{ab}) \geq 2(1+a)(1+b)
\end{equation}

After we expand each term and simplify, we get the following terms:

\begin{gather}
 2\sqrt{ab} + a + b + b\sqrt{ab} + a\sqrt{ab} \geq 2a + 2b + 2ab \\
 2\sqrt{ab} - a - b + b\sqrt{ab} + a\sqrt{ab} - 2(\sqrt{ab})^2 \geq 0
\end{gather}

Factor out $\sqrt{ab} - 1$:

\begin{gather}
 (\sqrt{ab} - 1)(a+b) + 2(\sqrt{ab} - (\sqrt{ab})^2) \geq 0 \\ 
 (\sqrt{ab} - 1)(a+b) + 2(\sqrt{ab})(1 - \sqrt{ab}) \geq 0 \\
 (\sqrt{ab} - 1)(a+b) - 2(\sqrt{ab})(\sqrt{ab} - 1) \geq 0 
\end{gather}

Now divide both sides by the common factor:

\begin{gather}
 (a + b) - 2\sqrt{ab} \geq 0 \\
 (a + b) \geq 2\sqrt{ab} \\
 \frac{a + b}{2} \geq \sqrt{ab}
\end{gather}

However, this is exactly the Arithmetic-Geometric Mean Inequality which holds when $a \geq 1$ and $b \geq 1$.

Thus, the inequality is proven.



\end{document}
