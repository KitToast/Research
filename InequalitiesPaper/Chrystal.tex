\documentclass[11pt]{article}
\usepackage{amsmath,textcomp,amssymb,geometry,graphicx,enumerate}

\def\Name{Edward Kim}  % Your name

%for code%
\usepackage{listings}
\lstset{language=python}


\title{Proof of Chrystal's Inequality}
\author{\Name}

\textheight=9in
\textwidth=6.5in
\topmargin=-.75in
\oddsidemargin=0.25in
\evensidemargin=0.25in
\DeclareMathSizes{11}{19}{13}{9}

\begin{document}
\maketitle

\textbf{Proof:}

We first note a couple things about the hypothesis. Firstly, \newline
\begin{equation}
  \prod_{i} a_i = \l^n \implies (\prod_{i} a_i)^{-n} = \l 
\end{equation}


We also know that the above expression is not undefined since $a > 0$. \newline

Now we shall invoke Theorm 40 (pg 39) to prove the claim.

Let our sets be the following: $(1, a_1), (1, a_2), ... , (1, a_n)$.
When we invoke the theorm, we get the following inequality:
\begin{equation}
 (1+a_1)^{-n}(1+a_2)^{-n}...(1+a_n)^{-n} > (\prod^{n}1^{-n} + \prod_{i}a_i^{-n}) \implies
 ((1+a_1)(1+a_2)...(1+a_n))^{-n} > (\prod^{n}1^{-n} + \prod_{i}a_i^{-n})
\end{equation}

However, by our observation above,

\begin{equation}
 ((1+a_1)(1+a_2)...(1+a_n))^{-n} > (1 + \l)
\end{equation}

After we exponentiate both sides by $n$, we get

\begin{equation}
  (1+a_1)(1+a_2)...(1+a_n)> (1 + \l)^{n}
\end{equation}

We know that the inequality is preserved since $a > 0$. 
Also if all $a_1 = a_2 = ... = a_n$, then $\frac{1}{a_1} = \frac{1}{a_2} = ... = \frac{1}{a_n}$. This would contradict our requirements for the 
validity of Theorm 40. So not all $a_i$'s can take one the same value.




\end{document}
