\documentclass[12pt]{amsart}
\usepackage{amsfonts}
\usepackage{amsmath}
\usepackage[a4paper, top=2.5cm, bottom=2.5cm, left=2.2cm, right=2.2cm]%
{geometry}
\usepackage{times}
\usepackage{amsmath}
\usepackage{amssymb}
\usepackage{tikz}
\usetikzlibrary{cd}

\newtheorem{theorem}{Theorem}[section]
\newtheorem{corollary}{Corollary}[theorem]
\newtheorem{lemma}[theorem]{Lemma}
\newtheorem{proposition}{Proposition}[theorem]


\newtheorem{definition}[theorem]{Definition}
\newtheorem{example}[theorem]{Example}
\newtheorem{xca}[theorem]{Exercise}

\newtheorem{remark}[theorem]{Remark}
\numberwithin{equation}{section}

\newtheorem{notation}[theorem]{Notation}

\begin{document}

\title{Cohomology of Connected Compact Lie Groups}
\author{Edward Kim}
\date{\today}
\maketitle
\tableofcontents

\section{De Rham Cohomology}
We rapidly review some definitions and concepts from de Rham cohomology before introducing the main topics. Recall that for a $n$-dimensional smooth manifold $M$, we define $\Omega^p(M)=\Gamma(M,\bigwedge^p(T^*M))$ to be the \emph{$p^{th}$ differential forms} on $M$ i.e smooth sections of the $p$-alternating tensor bundle. The direct sum of these vector spaces:
  $$\Omega^*(M) = \bigoplus_{p = 0}^{n} \Omega^p(M)$$
along with the wedge product $\wedge: \Omega^n(M) \times \Omega^m(M) \rightarrow \Omega^{n+m}(M)$ turns $\Omega^*(M)$ into an associative, anticommutative graded $C^{\infty}(M)$-algebra. Furthermore, there exists an \emph{exterior derivative}: $d: \Omega^{n}(M) \rightarrow \Omega^{n+1}(M)$ defined as follows on basis forms:
$$ d(fdx^1 \wedge dx^2 \wedge \cdots \wedge dx^n) = df \wedge dx^1 \wedge dx^2 \wedge \cdots \wedge dx^n$$
We will frequently use the global formula for the exterior derivative on $\Omega^p(M)$:
\begin{align} \label{globalexter}
  d\omega(X_1,\cdots, X_{p+1})
  & = \sum_{i = 1}^{p+1} (-1)^i X_i\omega(X_1,\cdots,\widehat{X_i},\cdots,X_{p+1}) \\
  & + \sum_{i < j} (-1)^{i+j} \omega([X_i,X_j],X_1,\cdots, \widehat{X_i},\cdots,\widehat{X_j},\cdots,X_{p+1})
\end{align}
where $X_1,\cdots,X_{p+1}$ are smooth vector fields on $M$.
Now it turns out that $d \circ d = 0$, defining the following cochain complex:
%
\begin{definition}
  The $p^{th}$ \emph{de Rham cohomology group} is defined to be the quotient groups
  \begin{equation}
    H^p(M) = Z^p(M) / B^p(d: \Omega^{p-1}(M) \rightarrow \Omega^{p})
  \end{equation}
  where
  \begin{align*}
    & Z^p(M) = \text{Ker}\left(d: \Omega^p(M) \rightarrow \Omega^{p+1} \right) \\
    & B^p(M) = \text{Im}\left(d: \Omega^p(M) \rightarrow \Omega^{p+1} \right)
  \end{align*}
\end{definition}
For $\omega \in \Omega^p(M)$, recall that $\omega$ is called \emph{closed} if $d\omega = 0$ and is called \emph{exact} if there exists $\eta \in \Omega^{p-1}(M)$ such that $\omega = d\eta$. Thus, $Z^p(M)$ denotes the closed $p$-forms on $M$ while $B^p(M)$ denotes the exact $p$-forms on $M$. The following proposition shows that smooth maps induce homomorphisms between the cohomology groups via their pullbacks:
%
\begin{proposition}
  Let $F:M \rightarrow N$ be a smooth map. The pullback $F^*: \Omega^p(N) \rightarrow \Omega^p(M)$ takes forms from $Z^p(N)$ into $Z^p(M)$ and $B^p(N)$ into $B^p(M)$, inducing a homomorphism between cohomology groups $H^p(N) \rightarrow H^p(M)$.
\end{proposition}
%
\begin{proof}
  If $\omega \in \Omega^p(N)$ is closed, then $dF^*\omega = F^*d\omega = 0$ since the exterior derivative commutes with pullbacks . Similarly, if $\omega = d\eta$ for some $\eta \in \Omega^{p-1}(N)$, then $F^*\omega = F^*d\eta = dF^*\eta$. This shows the proposition with the induced cohomology map being defined as:
  $$ F^*[\omega] = [F^*\omega], \quad \omega \in \Omega^p(N) $$
\end{proof}
This further shows that diffeomorphisms between manifolds induce isomorphisms between their respective cohomology groups. For more information about background topics, see \cite{lee}.

\section{Left-invariant Forms}
We base a good bit of our presentation on the articles by Fok and Zhang (\cite{fok},\cite{zhang}). The discussions on invariant forms are based on the treatment in \cite{chevalley}. The original paper detailing these arguments is \cite{reeder}.

We begin by deeming $G$ to be a connected compact Lie group. First, we start with the definition of a left-invariant
%
\begin{definition}
  Let $\omega \in \Omega^p(G)$. Then $\omega$ is said to be \emph{left-invariant} if
  \begin{equation}
    L_g^*\omega = \omega, \quad g \in G
  \end{equation}
\end{definition}
%
\noindent We denote $\Omega_L^p(G)$ to be the subspace of left-invariant $p$-forms on $G$.
%
\begin{lemma}
  Let $\omega \in \Omega_L^p(G)$. Then $d\omega \in \Omega_L^{n+1}$
\end{lemma}
\begin{proof}
  For all $g \in G$:
  $$L^*_gd\omega = dL^*\omega = d\omega$$
\end{proof}
\noindent This shows that $\Omega_L^{*}(G)$ form a cochain complex under $d$. Define $H^p_L(G)$ be cohomology of $\Omega^p_L(G)$. Furthermore, note that since pullbacks distribute over the wedge product:
$$ L^*_g(\omega_1 \wedge \omega_2) = L^*_g\omega_1 \wedge L^*_g\omega_2  = \omega_1 \wedge \omega_2$$
This shows that when $\omega_1 \in \Omega_L^n(G)$ and $\omega_2 \in \Omega_L^m(G)$, then $\omega_1 \wedge \omega_2 \in \Omega_L^{n+m}(G)$. In fact, the wedge product becomes a well-defined map: $\wedge: H^n_L(G) \times H^{m}_L(G) \rightarrow H^{n+m}_L(G)$ given by $$([\omega_1] ,[\omega_2]) \mapsto [\omega_1 \wedge \omega_2]$$
This endows a ring structure on $H^{*}_L(G)$ and an identical argument shows that $H^*(G)$ is also a ring. Now as subspaces, there exists an inclusion map $i: \Omega_L^p(G) \hookrightarrow \Omega^p(G)$ which induces a cohomology map between complexes $i': H^*_L(G) \rightarrow H^*(G)$. In fact, $i'$ is a ring isomorphism:
%
\begin{theorem} \label{leftrep}
$i'$ is a ring isomorphism
\end{theorem}
%
\begin{proof}
  To show surjectivity, given a form $\omega \in \Omega^p(G)$, there is an invariant form $\omega' \in \Omega^p_L(G)$ defined as
  %
  \begin{equation}
    \omega' = \int_G L^*_g\omega \; dg
  \end{equation}
  Recall that a similar trick is used to create an invariant inner product from an existing inner product on some representation space. Then
  %Arugment zhang
  To show injectivity: let $\omega \in \Omega_L^p(G)$ such that $\omega = d\eta$ for some $\eta \in \Omega^{p-1}(G)$. Then
  $$\omega = \omega' = \int_G L^*_g\omega \; dg = \int_G L^*_g d\eta \; dg = d \int_G L^*_g \eta \; dg = d\eta' $$
  The first equality is by left-invariance of $\omega$. Then $\omega$ must be an exact in respect to cochain complex $\Omega^*_L(G)$, showing injectivity.
\end{proof}
%
Theorem \ref{leftrep} states that the classes of the de Rham complex of $G$, $H^p(G)$ can be represented through its left-invariant counterparts $H^p_L(G)$. The additional structure placed on the left-invariant forms on $G$ give rise to an isomorphism between alternating forms on the Lie algebra of $G$, $\mathfrak{g}$  and the left-invariant forms
\begin{lemma}
Define
      \begin{gather}
        \psi: \Omega_L^{*}(G) \rightarrow \bigwedge^* \mathfrak{g}^* \\
        \omega \mapsto \{\omega\} := \omega\vert_{\wedge^n T_eG}
      \end{gather}
or explicitly:
$\{\omega\}(X_1,\cdots X_n) = \omega_e((X_1)_e, \cdots, (X_n)_e)$. The map $\psi$ will be a ring isomorphism.
\end{lemma}
\begin{proof}
  Showing that $\psi$ is a ring homomorphism just involves reasoning about the wedge product. To show surjectivity, take some $\mu \in \bigwedge^n\mathfrak{g}^*$ and define the $p$-form
  $$ \omega_g((X_1)_g,\cdots,(X_n)_g) = \mu((L_{g^{-1}})_*(X_1)_g, \cdots, (L_{g^{-1}})_*(X_n)_g) $$
  where $(L_{g^{-1}})_*(X_i)$ is the pushforward vector field of $X_i$ in respect to the diffeomorphism $L_{g^{-1}}$. To check that $\omega \in \Omega_L^n(G)$:
  \begin{align*}
    (L^*_g\omega)_h((X_1)_h, \cdots ,(X_n)_h)
    & = \omega_{gh}((L_g)_*(X_1)_h,\cdots, (L_g)_*(X_n)_h) \\
    & = \mu((L_{h^{-1}g^{-1}}L_g)_*(X_1)_h, \cdots (L_{h^{-1}g^{-1}}L_g)_*(X_n)_h) \\
    & = \mu((L_{h^{-1}})_*(X_1)_h), \cdots, (L_{h^{-1}})_*(X_n)_h) \\
    & = \omega_h((X_1)_h, \cdots, (X_n)_h)
  \end{align*}
  so $L^*_g\omega = \omega$ as desired. As for injectivity, suppose we had $\alpha \in \Omega^n_L(G)$ such that $\{\alpha\} = 0$. By a similar line of thinking as above:
  \begin{align*}
    \alpha_h((X_1)_h,\cdots,(X_n)_h)
    & = (L^*_h\alpha)_e((L_{h^{-1}})_*(X_1)_h, \cdots, (L_{h^{-1}})_*(X_n)_h) \\
    &= \alpha_e((L_{h^{-1}})_*(X_1)_h, \cdots, (L_{h^{-1}})_*(X_n)_h) \\
    & = 0
  \end{align*}
\end{proof}

There is actually a stronger property based on this ring isomorphism $\psi$. We will now show that there exists a homomorphism $\delta: \bigwedge^n\mathfrak{g}^* \rightarrow \bigwedge^{n+1}\mathfrak{g}^*$ which commutes with the $\psi$, turning $\bigwedge^*\mathfrak{g}^*$ into a cochain complex.
%
\begin{lemma}
There exist a linear map $\delta$ which makes the below diagram commute:
\begin{figure}[h!]
  \centering
  %
  \begin{tikzcd}
    \Omega_L^n \arrow[r, "d"] \arrow[d, "\psi"] &
    \Omega_L^{n+1} \arrow[d, "\psi"] \\
    \bigwedge^n\mathfrak{g}^* \arrow[r, "\delta"] &
    \bigwedge^{n+1}\mathfrak{g}^*
  \end{tikzcd}
\end{figure}

with the explicit form
\begin{equation}
  \delta\omega(\xi_1,\cdots, \xi_{n+1}) = \sum_{i < j} \omega([\xi_i,\xi_j], \xi_1,\cdots,\widehat{\xi_i},\cdots,\widehat{\xi_j},\cdots, \xi_{n+1})
\end{equation}
\end{lemma}
%
\begin{proof}
  Let us first take the $\xi_1,\cdots,\xi_{n+1}$ which are tangent vectors living in $T_eG$ and extend them to left-invariant vector fields on $G$ via a pushforward, $(X_i)_g = (L_g)_*(\xi_i)$. Expand $\psi \circ d$ as follows:
  \begin{align*}
    (\psi d\omega)(\xi_1,\cdots, \xi_{n+1})
    & = d\omega(X_1,\cdots,X_{n+1}) \\
    & = \sum_{i < j} (-1)^{i+j} \omega([X_i,X_j],X_1,\cdots,X_i, \cdots, \widehat{X_i}, \cdots, \widehat{X_j}, \cdots,X_{n+1}) \\
    & = \sum_{i < j} (-1)^{i+j} \{\omega\}([(X_i)_e,(X_j)_e],(X_1)_e,\cdots,(X_i)_e, \cdots, \widehat{X_i}, \cdots, \widehat{X_j}, \cdots,(X_{n+1})_e) \\
    & = (\delta \psi \omega)(\xi_1,\cdots,\xi_{n+1})
  \end{align*}
  The first equality follows since the restriction is $(X_i)_e = \xi$ by construction. The second equality arises from the left-invariance of $\omega$ and the $X_i$. Recall the global formula in Equation \ref{globalexter} and the terms:
  \begin{align*}
     X_i\omega(X_1,\cdots,\widehat{X_i},\cdots,X_{n+1})
     & = \mathcal{L}_{X_i}(\omega(X_1,\cdots,\widehat{X_i},\cdots,X_{n+1}) \\
    & = (\mathcal{L}_{X_i})(\omega(X_1,\cdots,\widehat{X_i},\cdots,X_{n+1})
    + \sum_{j = 0, j \neq i } \omega(X_1,\cdots [X_i,X_j], \cdots X_{n+1}) \\
    & = \sum_{j = 0, j \neq i } \omega(X_1, \cdots, \widehat{X_i}, \cdots, [X_i,X_j], \cdots X_{n+1})
  \end{align*}
  since $\mathcal{L}_{X_i}\omega = 0$ and $\mathcal{L}_{X_i}X_i = [X_i,X_i] = 0$. Taking the total sum reveals that:
  \begin{equation}
    \sum_{i=0}^{n+1}X_i\omega(X_1,\cdots,\widehat{X_i},\cdots,X_{n+1}) = 0
  \end{equation}
  The third equality is once again just from the restrictions of the $X_i$ and the definition of $\psi$.
\end{proof}
\noindent The lemma shows that the $\bigwedge^*\mathfrak{g}^*$ under the $\delta$ map is a cochain complex with $\psi$ inducing a cochain complex isomorphism.
\begin{theorem}
  $H^*_L(G) \cong H^*(\bigwedge^*\mathfrak{g}^*,\delta) $
\end{theorem}
\noindent $H^*(\mathfrak{g}) = H^*(\bigwedge^*\mathfrak{g}^*,\delta)$ is called the \emph{Lie algebra cohomology} of $\mathfrak{g}$. We have shown that for connected compact groups, its Lie algabra cohomology isomorphic to its left-invariant cohomology groups.

\section{Bi-invariant Forms}
We can extend these results to the so-called \emph{Bi-invariant classes} of $G$
%
\begin{definition}
  Let $\omega \in \Omega^n(G)$. $\omega$ is bi-invariant if $\omega \in \Omega_B^n(G) = \Omega_L^n(G) \cap \Omega_R^n(G)$. That is:
  \begin{align*}
    & L^*_g\omega = \omega \\
    &  R^*_g\omega = \omega
  \end{align*}
  for all $g \in G$
\end{definition}
%
The use of the average to discover bi-invariant cohomology representatives in $[\omega]$ also apply:
\begin{equation}
  \omega' = \int_{G}\int_{G} L^*_{g}R^*_{g}\omega \; dg dg
\end{equation}
A similar argument to Theorem \ref{leftrep} yields that the inclusion of subspaces $\Omega_B^n(G) \hookrightarrow \Omega^n(G)$ is also a ring isomorphism
%
\begin{theorem}
  $H_B^{*}(G) \cong H^*(G)$ as rings
\end{theorem}

\bibliographystyle{amsplain}
\bibliography{biblio}

\end{document}
